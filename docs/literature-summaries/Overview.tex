\documentclass{article}
\usepackage{titlesec}
\titleformat{\section}[block]{\Large\bfseries\filcenter}{}{1em}{}
\begin{document}
\section{Main papers}

\subsection*{A Distributed Consensus Protocol for Synchronization in Wireless Sensor Networks (2007)}
One of several AST papers. Good walkthrough of the basic details of time synchronization.

\subsection*{Average TimeSynch: A consensus-based protocol for clock synchronization in wireless sensor networks (2010)}
Main AST paper. References "A Distributed Consensus..." paper as preliminary work on AST. Provides experimental comparison of AST and the earlier FTSP protocol.

\subsection*{SATS: Secure Average-Consensus-Based Time Synchronization in Wireless Sensor Networks (2013)}
Improvement on AST, making it secure to malicious actors/nodes who use message manipulation to mess with the algorithm. This should be the algorithm we implement and compare (?)

\subsection*{Time Synchronization in WSNs: A Maximum-Value-
Based Consensus Approach (2013)}
Proposes MTS, which doesn't handle communication delay, and WMTS which does. WMTS seems to take over the work of MMTS? TODO: figure out MMTS vs. WMTS

\subsection*{Study of consensus-based time synchronization in wireless
sensor networks (2013)}
Compares AST and MTS experimentally, and proposes MMTS which is a modification of MTS that performs better in real life situations. This should be the algorithm we implement and compare (?)

\vspace{4em}
\section{Historical papers}

\subsection*{An Election Algorithm for a Distributed Clock Synchronization Program (1985)}
Describes the TEMPO synchronization algorithm. Not really useful except for historical perspective.

\subsection*{The Accuracy of the Clock Synchronization Achieved by TEMPO in Berkeley UNIX 4.3BSD (1989)}
Not super useful (?)

\subsection*{Fine-Grained Network Time Synchronization using Reference Broadcasts (2002)}
Presents the RBS protocol, which is intended as an improvement over NTP.

\subsection*{The Flooding Time Synchronization Protocol (2004)}
Proposes FTSP, which predates AST.

\vspace{4em}
\section{Theoretical papers}

\subsection*{Time, Clocks, and the Ordering of Events in a Distributed System (1978)}
Early paper on the meta-theory of logical clocks.

\subsection*{The Byzantine Generals Problem (1982)}
Introduces the Bynzantine Generals Problem.

\subsection*{Byzantine Clock Synchronization (1984)}
Short overview of the Byzantine problem, and 3 short failure-tolerant algorithms.

\subsection*{Virtual Time and Global States of Distributed Systems (1988)}
Talks about modelling time and clocks in a distributed system.

\subsection*{Fault-Tolerant Clock Synchronization in Distributed Systems (1990)}
In-depth overview of Byzantine problem and algorithms. Long newspaper-like article from NASA. Not super useful (?)

\subsection*{Implementing Fault-Tolerant Services Using the State Machine Approach: A Tutorial (1990)}
Talks about making server $\to$ client protocols fault-tolerant by replicating the servers/clients using state machines. Not super useful (?)

\end{document}