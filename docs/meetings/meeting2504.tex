\documentclass{article}
\usepackage[utf8]{inputenc}
\usepackage[]{hyperref}

\title{Meeting 25/4}
\author{Project Group 26}
\date{\today}

\begin{document}

\maketitle

\section{Meeting}

Specified problem statement, chosen specific algorithms ATS and MMTS. That's what we are analyzing and implementing. Started planning what to present at mid-way presentation. Eg. enplaning our progress and what we found difficult. We made a template, of the structure. In the presentation, we need to quantify why we chose those algorithms. We chose those algorithms partly because of wireless devices, they are also newer algorithms.

Regarding comparison we can extract comparisons from the different papers. Regarding literature it's cool, many papers. 

We are looking into splitting up problem questions and dividing them among us. Sounds good to Larisa.

Regarding implementation one person will be doing the implementation of the network implementation, and then maybe another person could implement another algorithms. To Larisa it sounds more concrete, which is great.

We can always meet on Skype. We know when the mid-way presentation is. Last years mid-way presentation, in this we should present to someone, then we will get feedback. No need to prepare for other peoples presentations. (We of course need to prepare for our own.)

\end{document}
