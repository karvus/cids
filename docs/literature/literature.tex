\documentclass[a4paper,12pt]{article}
\usepackage[utf8]{inputenc}
\usepackage[T1]{fontenc}

\usepackage[style=alphabetic,sorting=debug]{biblatex}
\DeclareFieldFormat{labelalpha}{\thefield{entrykey}}
\DeclareFieldFormat{extraalpha}{}
\addbibresource{../report/report.bib}
 
\begin{document}

\section{Consensus-based}

\begin{itemize}
\item Average TimeSync
  \begin{itemize}
  \item \cite{SchenatoGamba07}
  \item \cite{LucaFiorentin11}
  \end{itemize}
\end{itemize}

\begin{itemize}
\item SATS - Secure Average-Consensus-Based Time Synchronization
  \begin{itemize}
  \item \cite{HeChengShiChen13}
  \end{itemize}
\end{itemize}

\begin{itemize}
\item MMTS - about adopting both maximum consensus and minimum consensus
  \begin{itemize}
  \item \cite{HeLiChenCheng13}
  \end{itemize}
\end{itemize}

\begin{itemize}
\item MTS - Maximum Time Synchronization
  \begin{itemize}
    \item \cite{HeChengShiChen14}
  \end{itemize}
\end{itemize}

\begin{itemize}
\item 
\end{itemize}



\section{Non-consensus}

\begin{itemize}
\item \cite{ElsonGirodEstrin02} RBS - Reference Broadcast Synchronization
\item Berkeley
  \begin{itemize}
  \item \cite{Gusella89} The main paper about the Berkeley protocol
  \item \cite{GusellaZatti85} 85 paper on Berkeley protocol (Also by Gusella)
  \end{itemize}
\item FTSP - The Flooding Time Synchronization Protocol, an elected node maintains the global time (wireless sensor networks).
    \begin{itemize}
        \item \cite{Maroti04} 
    \end{itemize}
\item TPSN - Timing-sync Protocol for Sensor Networks,
    \begin{itemize}
        \item 
    \end{itemize}
\end{itemize}



\section{Other}

\begin{itemize}
\item \cite{Schneider90} Tutorial about implementing distributed state-machines (not consensus based AFAICT)
\item \cite{Lamport78} Logical clocks, and using them to implement ``real'' time synchronization
\item \cite{Lamport82} Seminal work about the Byzantine Generals Problem, as a metaphor for describing a fault-mode in distributed systems
\item \cite{Ramanathan90} Description of a hybrid software/hardware scheme for synchronization
\item \cite{Mattern89} Describes a way to use vectors as a way of representing "virtual time"
\end{itemize}

\printbibliography

\end{document}