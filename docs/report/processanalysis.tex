\documentclass[12pt]{article}
\usepackage[utf8]{inputenc}
\usepackage{amsmath}

\usepackage{amssymb}

\title{Process Analysis: Finding Consensus in Distributed Systems}
\author{Johan Ringmann Fagerberg -
  Marcus Møller\\
  Lucas Olai Jarlkov Olsen - 
  Peter Heilbo Ratgen \\
  Thomas Stenhaug 
  \\
  Advisor: Larisa Safina}
\begin{document}
\maketitle
\section{Idea Phase (Week 14-15)}
Our very first meeting as a group involved brainstorming a specific topic to research relevant to the overall project. We discussed various topics that utilize some sort of consensus in distributed systems. These were our initial ideas:

\begin{itemize}
    \item Load Balancing
    \item Time Synchronization
    \item Banking Transaction
    \item Peer-to-peer video games
    \item Block-chain
\end{itemize}

However, we agreed that these topics were all too broad. We would have to choose one and narrow it down further to a specific problem relating to one of these consensus fields. In the idea phase, our supervisor introduced us to a technique for breaking down the complexity of a problem. It involved studying a problem and attempting to break an otherwise too complicated task/problem down into smaller problems recursively, until a surmountable problem presented itself. This work phase also involved agreeing on how we would manage the project and distribute work. We also quickly started our research the same week, being provided with a few sources of literature to introduce us to the synchronization problem in depth. The second week we decided on clock synchronization as our consensus problem. We discussed a possible implementation of the algorithms and began brainstorming a problem statement.

\section{Research phase (Week 16-17)}
We transitioned into the research phase after we specified we wanted to work on the problem of time synchronization. Then we begun writing out our problem statement. While this was ongoing, we searched  for literature and we struck upon the consensus algorithms ATS and MMTS. Also looked at several different non-consensus algorithms we might incorporate into the report. 
Later in the same week (17) we finished our preliminary abstract for use at the mid-way seminar the week after. We also deciding on creating a simulated WSN to test out ATS and MMTS.

\section{Implementation and report (Week 18-21)}

In week 18, after the mid-way seminar. At the mid-way seminar we didn't get much criticism, and thus didn't get much to improve or work on. However it was great to have a milestone to work towards. In this week we each assumed responsibility for at portion of the report or implementation.

Much of the work on the report and implementation was done individually, which served us well, and we kept up communication especially as the deadline neared.  

\section{Group meetings and communication}
Throughout the idea and research phases, we would meet frequently and discuss our ideas and readings for the report. During the implementation and report phase, however, we would not meet as often, but we would distribute tasks that we would each finish before the next meeting. In between meetings, we would coordinate work in our online group chat.

\section{Conclusion}
Overall our process working on this project has been pretty satisfying. We had a few problems along the way, we struggled to make decisions on what to incorporate in the project, a few of our milestones collided with deadlines in other courses and communication with the supervisor sometimes led to confusion.


\end{document}