\documentclass[a4paper,12pt]{article}
\usepackage[utf8]{inputenc}
\usepackage[T1]{fontenc}
\usepackage[sorting=nyt]{biblatex}
\addbibresource{report.bib}
\usepackage[normalem]{ulem}
\usepackage{hyperref}
\usepackage{libertine}
\usepackage[scaled=0.89]{inconsolata}
\usepackage[showframe=false,bottom=6em,head=8em]{geometry}
\usepackage[iso,danish]{isodate}
\usepackage{fancyvrb}
\usepackage{fancyhdr}

\pagestyle{fancy}
\fancyhf{}
\lhead{FF501 Consensus in Distributed Systems}
\rhead{\today}
\chead{}
\lfoot{J. R. Fagerberg, M. Møller, L. O. J. Olsen, P. H. Ratgen, T. Stenhaug}
\cfoot{}
\rfoot{\thepage}
\renewcommand{\footrulewidth}{0pt}
\setcounter{secnumdepth}{2}
\setcounter{tocdepth}{2}

\date{\today}
\title{FF501 Consensus in Distributed Systems}
\author{
  Johan Ringmann Fagerberg \\
  Marcus Møller \\
  Lucas Olai Jarlkov Olsen \\
  Peter Heilbo Ratgen \\
  Thomas Stenhaug
}

\pagenumbering{roman}

\begin{document}

\maketitle

\setlength{\baselineskip}{1.44\baselineskip}

\begin{abstract}

  
  Nullam eu ante vel est convallis dignissim.  Fusce suscipit, wisi
  nec facilisis facilisis, est dui fermentum leo, quis tempor ligula
  erat quis odio.  Nunc porta vulputate tellus.  Nunc rutrum turpis
  sed pede.  Sed bibendum.  Aliquam posuere.  Nunc aliquet, augue nec
  adipiscing interdum, lacus tellus malesuada massa, quis varius mi
  purus non odio.  Pellentesque condimentum, magna ut suscipit
  hendrerit, ipsum augue ornare nulla, non luctus diam neque sit amet
  urna.  Curabitur vulputate vestibulum lorem.  Fusce sagittis, libero
  non molestie mollis, magna orci ultrices dolor, at vulputate neque
  nulla lacinia eros.  Sed id ligula quis est convallis tempor.
  Curabitur lacinia pulvinar nibh.  Nam a sapien.
  
\end{abstract}

\clearpage
\tableofcontents
\clearpage

\pagenumbering{arabic}
\setcounter{page}{1}

\section{Preface}

Nullam eu ante vel est convallis dignissim.  Fusce suscipit, wisi nec
facilisis facilisis, est dui fermentum leo, quis tempor ligula erat
quis odio.  Nunc porta vulputate tellus.  Nunc rutrum turpis sed pede.
Sed bibendum.  Aliquam posuere.  Nunc aliquet, augue nec adipiscing
interdum, lacus tellus malesuada massa, quis varius mi purus non odio.
Pellentesque condimentum, magna ut suscipit hendrerit, ipsum augue
ornare nulla, non luctus diam neque sit amet urna.  Curabitur
vulputate vestibulum lorem.  Fusce sagittis, libero non molestie
mollis, magna orci ultrices dolor, at vulputate neque nulla lacinia
eros.  Sed id ligula quis est convallis tempor.  Curabitur lacinia
pulvinar nibh.  Nam a sapien.

\section{Introduction (WIP)}

\subsection{Distributed systems}

A definition of \textit{distributed system} is
\begin{quote}
  A distributed system is a collection of autonomous computing
  elements that appears to its users as a single, coherent
  system.\cite{TanenbaumSteen06}
\end{quote}

The computing elements are typically called
nodes\cite{TanenbaumSteen06}, which is the term we will use within.
Nodes can be both hardware units, of software processes, while
\textit{users} might refer both to humans and software processes.

An example of a distributed system is multiplayer online game, where
the game environment is experienced as a single system, while there
are several nodes concerting the unified experience.  Another example
is the World Wide Web, which you can view as a large key-value store,
where Uniform Resource Locatiors (URL) are the key, and (typically)
HTML-pages are their value.  In the case of the WWW, both humans and
processes can considered users; indexing crawlers like Google are
examples of computer process users.  The type of distributed systems
we have studied, are \textit{Wireless Sensor Network}s (WSN), which is
discussed further below.

The type of distributed systems we have studied, are wireless sensor
networks.  Here, the nodes are small, cheap devices, equipped with the
ability to monitor environmental variables, like temperature,
humidity, sound pollution and so on. [cite?]

\subsection{Consensus}

Autonomous nodes that are to appear as a single system must share some
state.  For contrast, systems where there are master and slave nodes,
synchronization of shared data is achieved by pushing data from the
master to the slaves.  When nodes are ``peers'', consensus must be
reached by other means, and it must be reached in the face of faulty
or malicious nodes.

An example of faults, \textit{Byzantine faults}, were described in
Lamport, Shostak and Pease in a 1982 paper \cite{Lamport82}.  A group
of Byzantine generals, each general commanding a division, is
besieging a city.  They must reach an agreement on whether to all
attack or all retreat, and they can only communicate through messages.
There might be traiterous generals (faulty nodes), which send
different messages to different generals while loyal generals always
send the same message to all generals.

\subsection{Clock Synchronization}

There is not really a global clock in distributed systems.  In WSNs in
particular, it is too expensive to have an atomic clock at every node.

Depending on the application of the distributed system, different
``levels'' of synchronization might be chosen.  

Logical clocks \cite{Lamport78}, are chosen when it's only the strong
ordering of events within the system that is necessary.  If $c(x)$ is
a function of the system's ``clock'' at the point when event $x$.  Using
$a \rightarrow b$ to mean $a$ precedes $b$, then the strong clock condition is that
$a \rightarrow b \Rightarrow c(a) \rightarrow c(b)$

[TDSM, sync with atomic, prior protocols?]

\section{Theory}

Aliquam erat volutpat.  Nunc eleifend leo vitae magna.  In id erat non
orci commodo lobortis.  Proin neque massa, cursus ut, gravida ut,
lobortis eget, lacus.  Sed diam.  Praesent fermentum tempor tellus.
Nullam tempus.  Mauris ac felis vel velit tristique imperdiet.  Donec
at pede.  Etiam vel neque nec dui dignissim bibendum.  Vivamus id
enim.  Phasellus neque orci, porta a, aliquet quis, semper a, massa.
Phasellus purus.  Pellentesque tristique imperdiet tortor.  Nam
euismod tellus id erat.

\section{Methods}

Aliquam erat volutpat.  Nunc eleifend leo vitae magna.  In id erat non
orci commodo lobortis.  Proin neque massa, cursus ut, gravida ut,
lobortis eget, lacus.  Sed diam.  Praesent fermentum tempor tellus.
Nullam tempus.  Mauris ac felis vel velit tristique imperdiet.  Donec
at pede.  Etiam vel neque nec dui dignissim bibendum.  Vivamus id
enim.  Phasellus neque orci, porta a, aliquet quis, semper a, massa.
Phasellus purus.  Pellentesque tristique imperdiet tortor.  Nam
euismod tellus id erat.

\section{Results}

Lorem ipsum dolor sit amet, consectetuer adipiscing elit.  Donec
hendrerit tempor tellus.  Donec pretium posuere tellus.  Proin quam
nisl, tincidunt et, mattis eget, convallis nec, purus.  Cum sociis
natoque penatibus et magnis dis parturient montes, nascetur ridiculus
mus.  Nulla posuere.  Donec vitae dolor.  Nullam tristique diam non
turpis.  Cras placerat accumsan nulla.  Nullam rutrum.  Nam vestibulum
accumsan nisl.

\section{Discussion}

Lorem ipsum dolor sit amet, consectetuer adipiscing elit.  Donec
hendrerit tempor tellus.  Donec pretium posuere tellus.  Proin quam
nisl, tincidunt et, mattis eget, convallis nec, purus.  Cum sociis
natoque penatibus et magnis dis parturient montes, nascetur ridiculus
mus.  Nulla posuere.  Donec vitae dolor.  Nullam tristique diam non
turpis.  Cras placerat accumsan nulla.  Nullam rutrum.  Nam vestibulum
accumsan nisl.

\section{Conclusion}

Lorem ipsum dolor sit amet, consectetuer adipiscing elit.  Donec
hendrerit tempor tellus.  Donec pretium posuere tellus.  Proin quam
nisl, tincidunt et, mattis eget, convallis nec, purus.  Cum sociis
natoque penatibus et magnis dis parturient montes, nascetur ridiculus
mus.  Nulla posuere.  Donec vitae dolor.  Nullam tristique diam non
turpis.  Cras placerat accumsan nulla.  Nullam rutrum.  Nam vestibulum
accumsan nisl.

\nocite{*} % print all bibliography, remove when actual citations are in place
\printbibliography


\end{document}