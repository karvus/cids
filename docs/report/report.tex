\documentclass[a4paper,12pt]{article}
\usepackage[utf8]{inputenc}
\usepackage[T1]{fontenc}
\usepackage[sorting=nyt,citestyle=authoryear]{biblatex}
\addbibresource{report.bib}
\usepackage[normalem]{ulem}
\usepackage{hyperref}
\usepackage{libertine}
\usepackage[scaled=0.89]{inconsolata}
\usepackage[showframe=false,bottom=6em,head=8em]{geometry}
\usepackage[iso,danish]{isodate}
\usepackage{fancyvrb}
\usepackage{fancyhdr}

\pagestyle{fancy}
\fancyhf{}
\lhead{FF501 Consensus in Distributed Systems}
\rhead{\today}
\chead{}
\lfoot{J. R. Fagerberg, M. Møller, L. O. J. Olsen, P. H. Ratgen, T. Stenhaug}
\cfoot{}
\rfoot{\thepage}
\renewcommand{\footrulewidth}{0pt}
\setcounter{secnumdepth}{2}
\setcounter{tocdepth}{2}

\date{\today}
\title{FF501 Consensus in Distributed Systems}
\author{
  Johan Ringmann Fagerberg \\
  Marcus Møller \\
  Lucas Olai Jarlkov Olsen \\
  Peter Heilbo Ratgen \\
  Thomas Stenhaug
}

\pagenumbering{roman}

\begin{document}

\maketitle

\begin{abstract}

Clock synchronization is a widely known problem that has many applications. E.g. multiple clients in a distributed system must agree on a single time in order to successfully tackle a problem. The problem has been worked on since the birth of computer networks, and solutions have evolved over time in response to changing demands. In recent years, consensus based algorithms have gained traction with the propagation of Internet of Things, and several new algorithms have been developed as a result.

We will compare two recent consensus-based algorithms for clock synchronization in distributed systems, Average TimeSync (ATS) and Maximum Minimum Time Sync (MMTS), and implement them. The comparison will be based on the theoretical work from the algorithms' authors, our own analysis, and an implementation of both algorithms in virtual distributed networks.
\end{abstract}

\clearpage
\tableofcontents
\clearpage

\pagenumbering{arabic}
\setcounter{page}{1}

\section{Preface}

\section{Introduction}

Problem statement etc.

\section{Theory}

\subsection{A distributed system}

At first we need to define the term "distributed system". In \cite{Lamport78} a distributed system is defined as "a collection of distinct processes which are spatially separated and communicate to with one another by exchanging messages". These processes are also referred to as nodes. As stated in \cite{TanenbaumSteen06} these nodes either be software or hardware processes. These processes are programmed to achieve common goals, through working to together, and further passing messages. 


%Maybe elaborate a bit on this (although it shouldn't be "nice to have" knowledge, as per the report course).



\subsection{Time synchronization}

%Describe the time synchronization problem. Why is it necessary? Why is it not trivial?

    %Clock drift

In such a system where passing messages between nodes is essential and each node is separated. A single unified notion of time becomes an issue. Each node has a crystal, which keeps time. 

    %Time sync in 
 

%Describe the advantages and disadvantages of the different kinds of time synchronization strategies (non-consensus vs. consensus)
    
%Describe algorithms Average TimeSync and Modified Maximum Time Sync.

\section{Analysis}%Or comparison

Analyze the consensus-based algorithms Average TimeSync and Modified Maximum Time Sync. 

Compare the algorithms, their advantages and disadvantages, and their areas of application (through both implementation and theoretical analysis).

\section{Results}


\section{Discussion}

\section{Process analysis}%Not supposed to be in the report. Needs to be a separate document.

\section{Conclusion}



\nocite{*} % print all bibliography, remove when actual citations are in place
\printbibliography


\end{document}