\documentclass{beamer}
\usefonttheme[onlymath]{serif}
\usefonttheme{professionalfonts}
\usepackage[utf8]{inputenc}
\usepackage[style=alphabetic,sorting=debug]{biblatex}
\addbibresource{../report/report.bib}
 
%Information to be included in the title page:
\title{Consensus in Distributed Systems \\ Clock Synchronization}
\author{J. R. Fagerberg, M. Møller, L. O. J. Olsen,\newline P. H. Ratgen, T. Stenhaug}
\institute{IMADA}
\date{\today}
 
\begin{document}
 
\frame{\titlepage}
 
\begin{frame}
  \frametitle{Introduction}
  \begin{itemize}
  \item<1-> What is a distributed system?
    \begin{itemize}
    \item<2-> ``A distributed system is a collection of autonomous
      computing elements that appears to its users as a single
      coherent system.'' \footnote{Andrew S. Tanenbaum and Maarten van
        Steen. 2006. Distributed Systems: Principles and Paradigms
        (2nd Edition). Prentice-Hall, Inc., Upper Saddle River, NJ,
        USA.}
    \item<3-> ``Users'' refer to both software processes and humans.
    \item<4-> Examples
      \begin{itemize}
      \item<5-> WWW
      \item<6-> MMOs
      \item<7-> Mobile networks
      \item<8-> Wireless Sensor Networks (that we have chosen)
      \end{itemize}
    \end{itemize}
  \end{itemize}
\end{frame}

\begin{frame}
  \frametitle{Introduction}
  \begin{itemize}
  \item<1-> What is consensus in a distributed system?
    \begin{itemize}
    \item<2-> Problem: reliability/consistency in presence of faulty
      and/or malicious processes.
    \item<3-> Byzantine Generals problem
      \begin{itemize}
      \item<4-> Camped outside besieged city
      \item<5-> Attack or retreat: success iff loyal generals decide on same plan (consensus)
      \item<6-> Communicate by messenger only
      \item<7-> Traitorous generals analogue to faulty nodes
      \end{itemize}
    \end{itemize}
  \end{itemize}
\end{frame}

\begin{frame}
  \frametitle{Introduction}
  \begin{itemize}
  \item<1-> What is clock synchronization?
    \begin{itemize}
    \item<2-> Can be relative ordering of events in a distributed system
      (logical clocks, transactions)
    \item<3-> ...or that each node is coordinated within the system (TDSM)
    \item<4-> ...or absolute time, coordinated with an atomic clock.
    \end{itemize}
  \end{itemize}
\end{frame}

\begin{frame}
  \frametitle{Introduction}
  \begin{itemize}
  \item<1-> Synchronization protocols
    \begin{itemize}
    \item<2-> Master/slave (NTP, Berkeley, other variants)
    \item<3-> For WSNs; RBS, TPSN, FTSP etc.      
      \begin{itemize}
      \item<4-> tree-based in some sense
      \item<5-> sensitive to changing node topology
      \end{itemize}      
    \item<6-> Enter consensus-based
      \begin{itemize}
      \item<7-> Recall Byzantine Generals Problem
      \item<8-> We're looking at ATS and MMTS
      \end{itemize}
    \end{itemize}
  \end{itemize}
\end{frame}

\begin{frame}
\frametitle{Abstract}
Clock synchronization is a widely known problem that has many applications. E.g. multiple clients in a distributed system must agree on a single time in order to successfully tackle a problem. The problem has been worked on since the birth of computer networks, and solutions have evolved over time in response to changing demands. In recent years, consensus based algorithms have gained traction with the propagation of IoT, and several new algorithms have been developed as a result.\\
\vspace{1em}
We will compare two recent consensus-based algorithms for clock synchronization in distributed systems, ATS and MMTS, and implement them. The comparison will be based on the theoretical work from the algorithms' authors, our own analysis, and an implementation of both algorithms in virtual distributed networks.
\end{frame}

\begin{frame}{Problem statement}
\begin{itemize}
    \item<1-> Describe the time synchronization problem. Why is it necessary? Why is it not trivial? 
    \item<2-> Describe the advantages and disadvantages of the different kinds of time synchronization strategies (non-consensus vs. consensus)
    \item<3-> Describe and analyze the consensus-based algorithms ATS and MMTS.
    \item<4-> Compare the algorithms, their advantages and disadvantages, and their areas of application (through both implementation and theoretical analysis).
\end{itemize}

\end{frame}
 
\begin{frame}
    \frametitle{Choice of algorithms}
    \begin{itemize}
        \item Target: Wireless sensor networks
        \item Consensus-based
        \item By proxy - more recent algorithms
    \end{itemize}
\end{frame}
 
\begin{frame}{Algorithms - general}
    \begin{itemize}
        \item Hardware clocks modelled as linear functions $$ \tau_i(t) = \alpha_i t + \beta_i $$
        \item Hardware clocks can be modified into logical clocks
        \begin{align*}
            L_i(t) &= \hat{\alpha}_i \tau_i(t) + \hat{\beta}_i \\
                    &= \hat{\alpha}_i \alpha_i t + \hat{\alpha}_i \beta_i + \hat{\beta}_i
        \end{align*}
        \item Goal is to make $\hat{\alpha}_i \alpha_i \approx A$ and $\hat{\alpha}_i \beta_i + \hat{\beta}_i \approx B$, where $A$ and $B$ are constants in the network.\\
        That is, goal is to find values for $\hat{\alpha}_i$ and $\hat{\beta}_i$ for each node $i$ so that all logical clocks agree on what the time is.
    \end{itemize}
\end{frame}

\begin{frame}{Algorithms - Average Time-Sync}
    \begin{itemize}
        \item Incrementally finds $\hat{\alpha}_i$ and $\hat{\beta}_i$.
        \item Each node talks to a neighbor, and modifies own logical clock so that it is closer to the neighbors logical clock.
        \item How close it gets to the neighbors logical clock depends on algorithm configuration
        $$ x_{ours} \leftarrow \rho \cdot x_{ours} + (1 - \rho) \cdot x_{theirs}. $$
        \item This has been proven to converge eventually assuming no uncertainties (e.g. communication delay) \cite{LucaFiorentin11}
    \end{itemize}
\end{frame}

\begin{frame}{Algorithms - (Modified) Maximum Time-Sync}
    \begin{itemize}
        \item Quickly finds local consensus, incrementally finds global consensus.
        \item Each node talks to a neighbor, and modifies own logical clock to equal neighbors logical clock iff neighbors logical clock is greater than own logical clock.
        \item This has been proven to converge eventually assuming no uncertainties (e.g. communication delay) \cite{HeChengShiChen11}
        \item However, this doesn't work if communication delay is introduced. MMTS has been proposed as an alternative based on experimental results \cite{HeLiChenCheng13}
    \end{itemize}
\end{frame}

\begin{frame}{Implementation}
   \begin{itemize}
       \item Simulation of a distributed network
       \item Simulated time skew and/or delay on each node that the algorithms will have to deal with
       \item We will run tests with a number of different nodes and topologies to see how that influences the performance of the algorithms
       \item Measurement tools for comparing the different algorithms
       \item Functionally, the algorithms work quite similarly. Each node regularly updates their time, based on a comparison with the time of the neighbor.
   \end{itemize} 
\end{frame}
 
\begin{frame}{Process analysis}
    \begin{itemize}
        \item Gathering articles, material and information. Summarizing articles for later use in the report.
        \item Specifying a topic/problem was difficult due to initial lack of knowledge on the topic. 
        \item We've had fairly frequent meetings, 2-3 times a week, where the majority of our work was conducted. 
        \item Work is distributed according to proficiency. 
        e.g. those of us with more programming experience lead the implementation effort.
        \item Cooperation hasn't been an issue.
    \end{itemize}
\end{frame}

\begin{frame}{Literature}
    Consensus-based
    
        \begin{itemize}
            \item Average TimeSync
                \begin{itemize}
                \item \cite{SchenatoGamba07}
                    \item \cite{LucaFiorentin11}
                \end{itemize}
            \item SATS - Secure Average-Consensus-Based Time Synchronization - \cite{HeChengShiChen13}
            \item MMTS - about adapting both maximum consensus and minimum consensus- \cite{HeLiChenCheng13}
            \item MTS - Maximum Time Synchronization - \cite{HeChengShiChen14}
        \end{itemize}
\end{frame}

\begin{frame}{Literature}
    Non-consensus
    
        \begin{itemize}
        \item  RBS - Reference Broadcast Synchronization - \cite{ElsonGirodEstrin02}
        \item Berkeley
          \begin{itemize}
          \item \cite{Gusella89} The main paper about the Berkeley protocol
          \item \cite{GusellaZatti85} 85 paper on Berkeley protocol (Also by Gusella)
          \end{itemize}
        \item FTSP - The Flooding Time Synchronization Protocol - \cite{Maroti04} 
        \item TPSN - Timing-sync Protocol for Sensor Networks - \cite{GaneriwalKumarSrivastava03}
    \end{itemize}

\end{frame}
 
\end{document}