\documentclass{article}
\usepackage[utf8]{inputenc}

\title{Meeting 6/4}
\author{Project Group 26}
\date{April 2018}

\begin{document}

\maketitle

\section{Meeting}


This week  we can read the litterature, and we can find the problem we want to work with. She has litterature we can work with. Next time we can make a plan. We need to be focused, time is short. 

Last year she realised people have different skills. Everybody needs to do something, everybody need to do equal parts.

Hints: can choose any consensus algorithm, can compare performance between algorithms (or exsistenst algoritms). Compare implementation in different languages. Create a real world example and implement in this.

Can always write emails, she suggest meeting frequency once or twice a week. Keep together and discuss. Without brainstorming it's all about nothing, the first year project is about working together. 

Send questions to Larisa one day in advance. 

Last years team did time-sync. All the topics of the brainstorm seems great. 

Methods to constrain the problem? Create a high level goal, chop it into smaller parts. It's needed when we create a plan for the project. We will see what we need.

Remember to \textbf{send sharelatex link}. Remember to keep journal. It will influence the grade.

Look into the book given on in the project description, the latest 

\end{document}
